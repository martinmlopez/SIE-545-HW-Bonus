\documentclass[12pt]{article}
\usepackage{graphicx}
\usepackage{amsfonts}
\usepackage{fancyhdr}
\usepackage{comment}
\usepackage{amsmath}
\usepackage[a4paper, top=2.5cm, bottom=2.5cm, left=2.2cm, right=2.2cm]%
{geometry}
\usepackage{times}
\usepackage{amsmath}
\usepackage{changepage}
\usepackage{amssymb}
\usepackage{fixltx2e}
\usepackage{enumerate}
\usepackage{graphicx}
\usepackage{float}
\usepackage{booktabs}
\newtheorem{theorem}{Theorem}
\newtheorem{acknowledgement}[theorem]{Acknowledgement}
\newtheorem{algorithm}[theorem]{Algorithm}
\newtheorem{axiom}{Axiom}
\newtheorem{case}[theorem]{Case}
\newtheorem{claim}[theorem]{Claim}
\newtheorem{conclusion}[theorem]{Conclusion}
\newtheorem{condition}[theorem]{Condition}
\newtheorem{conjecture}[theorem]{Conjecture}
\newtheorem{corollary}[theorem]{Corollary}
\newtheorem{criterion}[theorem]{Criterion}
\newtheorem{definition}[theorem]{Definition}
\newtheorem{example}[theorem]{Example}
\newtheorem{exercise}[theorem]{Exercise}
\newtheorem{lemma}[theorem]{Lemma}
\newtheorem{notation}[theorem]{Notation}
\newtheorem{problem}[theorem]{Problem}
\newtheorem{proposition}[theorem]{Proposition}
\newtheorem{remark}[theorem]{Remark}
\newtheorem{solution}[theorem]{Solution}
\newtheorem{summary}[theorem]{Summary}
\newenvironment{proof}[1][Proof]{\textbf{#1.} }{\ \rule{0.5em}{0.5em}}

\newcommand{\Q}{\mathbb{Q}}
\newcommand{\R}{\mathbb{R}}
\newcommand{\C}{\mathbb{C}}
\newcommand{\Z}{\mathbb{Z}}

\begin{document}

\title{SIE 545: Fundamentals of Optimization \\Bonus Homework}
\author{Martin Manuel Lopez \\lopez9@email.arizona.edu \\Systems and Industrial Engineering}
\date{December 10, 2018}
\maketitle
\section{BSS 6.14}
    We have the following Primal problem:\\
        \begin{align*}
            &\min \quad f(x)\\
            &s.t.\\
            &\quad g(x) + s = 0\\
            &\quad h(x) = 0\\
            &\quad (x,s) \in X'\\
            &\quad X' = \{(x,s): x \in X, s \geq 0\}\\
        \end{align*}
    We will construct the dual using and utilizing the following dual problem:\\
        \begin{align*}
            &\theta(x, u) = \inf\{f(x) + u^T (g(x) + s) + v^T h(x) : (x,s) \in X' \}\\ 
        \end{align*}
    We separate the the slack vector $s$\\ 
        \begin{align*}
            &\theta(x, u) =  \inf\{f(x) + u^T (g(x)) + v^T h(x) : (x \in X' \} + \inf \{u^ts : s \geq 0 \}\\
        \end{align*}
    With the separation of the variable we notice that we get a piecewise function: \\
        \begin{align*}
            \begin{cases} 
                    \inf \{f(x) + u^Tg(x) + v^T h(x) : x\in X' & u \geq 0\\
                    - \infty & \text{otherwise}\\
                \end{cases}\\
        \end{align*}
\section{BSS 6.40}
    Consider the following: \\
        \begin{align*}
            &\min_{x \in X} \max_{y \in Y} \phi(x,y)\\\\
            &\quad \quad \text{and}\\\\
            &\max_{y \in Y} \min_{x \in X} \phi(x,y)\\
        \end{align*}
    X and Y are non-empty compact convex sets in $R^n$ and $R^m$, respectively, $\phi (x,y)$ is convex in x for any given y and concave for in y for any given x.\\\\
    a.) Show that $\min_{x \in X} \max_{y \in Y} \phi(x,y) \geq \max_{y \in Y} \min_{x \in X} \phi(x,y)$ without assuming convexity:\\
    Without having the assumption of convexity we define the following and prove convexity first:\\
        \begin{align*}
            &\min_{x \in X} \quad \phi(x,y) = g(y) \quad \forall y \in Y\\
            &\text{We prove convexity of }g(y)\\
            &\Longrightarrow\\
            &g(y) \leq g(\bar y) + \nabla g(\bar y)^T (y- \bar y)\\
        \end{align*}
    We have defined $g(x)$ as concave by the definition, $\phi (x,y)$ is differentiable within a compact set in $R^n$ for $x$. \\
    We then must prove concavity for y for any given x.\\
        \begin{align*}
            &\text{We let }h(x) \geq h(\bar x) + \nabla h(\bar x)^T (x - \bar x) \quad \forall x \in X\\
            &\text{Therefore, h(y) is concave}\\
        \end{align*}
    Now that we have defined convexity for the $\phi (x,y) \forall x,y$ we can prove $\min_{x \in X} \max_{y \in Y} \phi(x,y) \geq \max_{y \in Y} \min_{x \in X} \phi(x,y)$. \\
        \begin{align*}
            &\inf_{x \in X}\phi(x,y) \leq \phi(x,y) \leq \sup_{y \in Y} \phi (x,y)\\
            &\inf_{x \in X} \phi(x,y) = g(y)\\
            &\sup_{y \in Y}\phi(x,y) = h(x)\\
            &\Longleftrightarrow\\
            &g(y) \leq \phi(x,y) \leq h(x)\\
            &\Longleftrightarrow\\
            &\max_{y \in Y} g(y) \leq \phi (x,y) \leq \min_{x \in X} h(x)\\
            &\max_{y \in Y} \min_{x \in X} \phi(x,y) \leq \phi(x,y) \leq \min_{x \in X} \max_{y \in Y} \phi(x,y)\\
            &\Longleftrightarrow\\
            &\max_{y \in Y} \min_{x \in X} \phi(x,y) \leq \min_{x \in X} \max_{y \in Y} \phi(x,y)\\
        \end{align*}
    b.) Must show that $\min_{x \in X} \max_{y \in Y} \phi(x,y)$ is a convex function in x and $\max_{y \in Y} \min_{x \in X} \phi(x,y)$ is a concave function in y.\\
     Without having the assumption of convexity we define the following and prove convexity first:\\
        \begin{align*}
            &\min_{x \in X} \quad \phi(x,y) = g(y) \quad \forall y \in Y\\
            &\text{We prove convexity of }g(y)\\
            &\Longrightarrow\\
            &g(y) \leq g(\bar y) + \nabla g(\bar y)^T (y- \bar y)\\
        \end{align*}
    We have defined $g(x)$ as concave by the definition, $\phi (x,y)$ is differentiable within a compact set in $R^n$ for $x$. \\
    We then must prove concavity for y for any given x.\\
        \begin{align*}
            &\text{We let }h(x) \geq h(\bar x) + \nabla h(\bar x)^T (x - \bar x) \quad \forall x \in X\\
            &\text{Therefore, h(y) is concave}\\
        \end{align*}
    This was also shown in the first part of part A.\\\\
    c.) Show that $\min_{x \in X} \max_{y \in Y} \phi(x,y) = \max_{y \in Y} \min_{x \in X} \phi(x,y)$\\
    If we take the final part of part A:\\
        \begin{align*}
             &\max_{y \in Y} \min_{x \in X} \phi(x,y) \leq \min_{x \in X} \max_{y \in Y} \phi(x,y)\\
        \end{align*}
    We essentially have two conditions:\\
        \begin{align*}
            &\max_{y \in Y} \min_{x \in X} \phi(x,y) = 
            \begin{cases}
                     &= \min_{x \in X} \max_{y \in Y} \phi(x,y)\\
                    &< \min_{x \in X} \max_{y \in Y} \phi(x,y)\\
                \end{cases}\\
        \end{align*}
    We know that if we take the second case then it will go to $-\infty$:\\
        \begin{align*}
            &\max_{y \in Y} \min_{x \in X} \phi(x,y) = 
            \begin{cases}
                     &= \min_{x \in X} \max_{y \in Y} \phi(x,y)\\
                    &-\infty \quad \text{otherwise}\\
                \end{cases}\\
        \end{align*}
    Thus we have shown:\\
        \begin{align*}
             &\max_{y \in Y} \min_{x \in X} \phi(x,y) = \min_{x \in X} \max_{y \in Y} \phi(x,y)\\
        \end{align*}
\section{BSS 6.43}
    We have the following warehouse location problem stated mathematically:\\
        \begin{align*}
            &\min \quad \sum_{i=1}^{m} \sum_{j=1}^{k}c_{ij}x_{ij} + \sum_{i=1}^{m} f_iy_i\\
            &s.t.\\
            &\quad \sum_{j=1}^{k} x_{ij} \leq s_iy_i \quad \text{for } 1,...,m\\
            &\quad \sum_{i=1}^{m}x_{ij} \geq d_j \quad \text{for } 1,..,k\\
            &\quad 0 \leq x_{ij} \leq y_i \mi\{s_i,d_j\} \quad \text{for } i=1,...m ; j=1,...,k\\
            &\quad y_i = 0 \text{ or } 1 \quad \text{for } i=1,...,m\\
        \end{align*}
    a.) We must formulate a Lagrange Function for the problem and we assume that $y_i = 0$ that way we have the upper bound fixed at $x_{ij}$.\\
        \begin{align*}
            &\theta (x, u) = \min \quad \sum_{i=1}^{m} \sum_{j=1}^{k}c_{ij}x_{ij} + \sum_{i=1}^{m} f_iy_i + u_1 (\sum_{j=1}^{k} x_{ij} - s_iy_i) - u_2(\sum_{i=1}^{m}x_{ij} - d_j)\\
            &\quad \quad x \in X\\
        \end{align*}
    b.) We can now maximize the dual:
        \begin{align*}
            &\max \quad \theta(x,u)= \min \quad \sum_{i=1}^{m} \sum_{j=1}^{k}c_{ij}x_{ij} + \sum_{i=1}^{m} f_iy_i + u_1 (\sum_{j=1}^{k} x_{ij} - s_iy_i) - u_2(\sum_{i=1}^{m}x_{ij} - d_j)\\
            &\quad \quad x \in X\\
        \end{align*}
    c.) We must then illustrate a numerical example which can be dictated by $x \in X$ and the amount of iterations depends on $X$.
\end{document}